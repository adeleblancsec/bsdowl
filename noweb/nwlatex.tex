%
% D�finitions l2h pour LaTeX
%

% l2h substitution th &thorn;
% l2h argblock paragraph h4
% l2h argblock subparagraph h5

\makeatletter
\usepackage[cmyk,dvipsnames]{xcolor}
\definecolor{emphasize}{cmyk}{0,0,0,0.05}
\colorlet{example}{Bittersweet}

\usepackage{fancyvrb}

%
% Underscore mieux mieux
%
%\def\_{\/\vrule height 0.4bp depth 0pt width .333em\relax}

%
% Marquage des �l�ments du texte
%

\newcommand\fn[1]{\textsl{#1}}		% Function name
\newcommand\va[1]{\textit{#1}}		% Variable name
\newcommand\fa[1]{\textit{#1}}		% Formal argument
\newcommand\pa[1]{\textsl{#1}}		% Path

% l2h argblock fn em
% l2h argblock va var
% l2h argblock fa car
% l2h argblock pa em

\newcommand\fn@prototype[1]{\texttt{\textbackslash#1}}
\newcommand\fa@prototype[1]{\LA{}\textit{#1}\RA{}}

%
% Ajouter un point � la fin d'une suite de tokens, peut-�tre
%

\def\maybeaddadot#1{\maybeaddadot@A#1\relax.\relax\st@p}
\def\maybeaddadot@A#1.\relax#2\st@p{#1.}


%
% Structure
%

\newenvironment{formalparagraph}[1]%
  {%
    \par
    \bigbreak
    \noindent
    \begingroup
    \def\rmA{#1}%
    \ifx\rmA\empty\else
    \textit{#1.~---}\enspace
    \fi
    \endgroup
    \ignorespaces
  }%
  {\par\smallbreak}%

\newenvironment{remark}%
  {\begin{formalparagraph}{Remarque}}%
  {\end{formalparagraph}}%

\newenvironment{remarks}%
  {\begin{formalparagraph}{Remarques}\par\nobreak
  \ignorespaces
  \parindent\z@
  \parskip\smallskipamount}%
  {\end{formalparagraph}}%

\newenvironment{seealso}%
  {\let\maybeaddadot\relax
   \begin{formalparagraph}{Voir aussi:}\let\item\seealso@iA}%
  {\end{formalparagraph}}%

\def\seealso@A{\let\item\seealso@B}
\def\seealso@B{, }

%
% Exemples
%

% l2h envblock example pre
\DefineVerbatimEnvironment{example}%
  {Verbatim}%
  {%
    xleftmargin=\parindent,
    formatcom=\let\FancyVerbFormatLine\FancyExampleFormatLine
  }%


\def\FancyExampleFormatLine#1{\llap{\smash{$\vdots$}\enspace}\texttt{#1}}

\let\begin@example@O\example

\def\example{\begingroup\catcode`\^^M\active\example@M}

\newcommand\example@M[1][]{%
  \def\rmA{#1}%
  \ifx\rmA\empty
    \def\next{\begin@example@O}%
  \else
    \def\next{\begin@example@T{#1}}%
  \fi
  \expandafter\endgroup\next
}

\def\begin@example@T#1{%
  \par\bigbreak
  \noindent\textit{\maybeaddadot{#1}}\par\vskip-\medskipamount\nobreak%
  \begin@example@O
}%


%
% Alin�as sp�ciaux
%

\newcommand\paracommand[1]{%
  \par
  \vskip 4em plus 1em minus 1em
  \penalty -20
  \relax
  \begingroup
  \markslongfalse
  \marksouterfalse
  \marksprimitivefalse
  \toks0{}%
  \let\TeX@orig\TeX
  \def\long{\toksrappend\marks@l\to\toks0}%
  \def\outer{\toksrappend\marks@o\to\toks0}%
  \def\TeX{\toksrappend\marks@t\to\toks0}%
  \let\cs@orig\cs
  \def\cs##1{\gdef\currentcs{##1}\let\cs\cs@orig\cs{##1}}%
  \def\fa##1{\LA{}\textit{##1}\RA}%
  \dimen0=\hsize \advance\dimen0-6pt\relax
  \noindent\colorbox{emphasize}{\makebox[\dimen0]{#1\hfill\commandmarks}}\par
  \endgroup
  \nobreak
  \smallskip
  \noindent
  \ignorespaces
}

\def\commandmarks{%
  \the\toks0
  \let\next\relax
  \ifmarkslong\let\next\commandmarks@A\fi
  \ifmarksouter\let\next\commandmarks@A\fi
  \ifmarksprimitive\let\next\commandmarks@A\fi
  \next
}

\def\commandmarks@A{\footnotesize This is %
\ifmarksouter
  an outer \ifmarkslong long \fi
\else
  \ifmarkslong a long \fi
\fi
\ifmarksprimitive \TeX@orig\ primitive\else command\fi
.}

\def\marks@l{\markslongtrue}
\def\marks@o{\marksoutertrue}
\def\marks@t{\marksprimitivetrue}

\newif\ifmarkslong
\newif\ifmarksouter
\newif\ifmarksprimitive

\newcommand\paranamespace[1]{%
  \begin{formalparagraph}{}\em
  La biblioth�que r�serve l'espace de noms priv�~\texttt{#1}.%
  \end{formalparagraph}%
}

\def\parafakepredicate{%
  \begin{formalparagraph}{}\em
  Cette proc�dure est un faux pr�dicat. Les faux pr�dicats sont
  discut�s dans la section~XXX.
  \end{formalparagraph}%
}

\def\paraoutercatcode{%
  \begin{formalparagraph}{}\em
  Cette proc�dure manipule la table des %
  codes de cat�gorie des caract�res~\emph{(catcode table)}, il est %
  donc recommand� que les proc�dures l'appelant soient %
  marqu�es~\texttt{\string\outer}.%
  \end{formalparagraph}%
}
